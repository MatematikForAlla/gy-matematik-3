% $Id$
%
% Copyright 2013; Daniel Bosk <daniel@bosk.se>
%
% This work is licensed under the Creative Commons Attribution-ShareAlike 3.0 
% Unported license.  To view a copy of this license, visit URL
%
%   http://creativecommons.org/licenses/by-sa/3.0/.
%
\documentclass[a4paper,titlepage,reqno,final,twoside]{amsbook}
\usepackage{fixltx2e}
\usepackage[utf8]{inputenc}
\usepackage[T1]{fontenc}
\usepackage[german,english,ibycus,swedish]{babel}
\usepackage{graphicx}
\usepackage[hyphens]{url}
\usepackage{hyperref}
\usepackage{lettrine}
\usepackage{amssymb}
\usepackage{amsmath}
\usepackage{amsthm}
\usepackage{csquotes}
\usepackage[natbib,style=alphabetic,maxbibnames=99]{biblatex}
\usepackage{setspace}
\usepackage{nomencl}
\usepackage{marginnote}
\usepackage{subfig}
\usepackage{booktabs}
\usepackage[multiple]{footmisc}
\usepackage{verbatim}
\usepackage{varioref,prettyref}
\usepackage{ifdraft}
\usepackage{asymptote}

\ifdraft{
  \usepackage{fancyhdr}
  \usepackage{showlabels}
}{
  \usepackage[a4paper,rmargin=7cm,lmargin=2.5cm,bmargin=3.5cm]{geometry}
}

\addbibresource{matematik-3.bib}

\newtheorem{theorem}{Sats}[chapter]
\newtheorem{proposition}{Proposition}[chapter]
\newtheorem{lemma}{Lemma}[chapter]
\newtheorem{corollary}{Korollarium}[chapter]
\newtheorem{conjecture}{Förmodan}[chapter]
\theoremstyle{definition}
\newtheorem{axiom}{Axiom}[chapter]
\newtheorem{definition}{Definition}[chapter]
\newtheorem{example}{Exempel}[chapter]
\newtheorem{exercise}{Övning}[chapter]
\theoremstyle{remark}
\newtheorem{remark}{Anmärkning}[chapter]

\numberwithin{section}{chapter}
\numberwithin{equation}{chapter}

\renewcommand{\reftextbefore}{(föregående sida)}
\renewcommand{\reftextfacebefore}{(föregående sida)}
\renewcommand{\reftextafter}{(nästa sida)}
\renewcommand{\reftextfaceafter}{(nästa sida)}
\renewcommand{\reftextfaraway}[1]{(sidan \pageref{#1})}
\newrefformat{ch}{kapitel \ref{#1}}
\newrefformat{sec}{avsnitt \ref{#1}}
\newrefformat{app}{bilaga \ref{#1}}
\newrefformat{tab}{tabell \vref{#1}}
\newrefformat{tbl}{tabell \vref{#1}}
\newrefformat{fig}{figur \vref{#1}}
\newrefformat{ax}{axiom \ref{#1}}
\newrefformat{def}{definition \ref{#1}}
\newrefformat{thm}{sats \ref{#1}}
\newrefformat{pro}{proposition \ref{#1}}
\newrefformat{lem}{lemma \ref{#1}}
\newrefformat{cor}{korollarium \ref{#1}}
\newrefformat{ex}{exempel \ref{#1}}
\newrefformat{xrc}{övning \ref{#1}}
\newrefformat{remark}{anmärkning \ref{#1}}

\renewcommand{\qedsymbol}{Q.E.D.}
\newcommand{\N}{\mathbb{N}}
\newcommand{\Q}{\mathbb{Q}}
\newcommand{\R}{\mathbb{R}}
\newcommand{\Z}{\mathbb{Z}}
\newcommand{\powerset}{\mathcal{P}}
\newcommand{\U}{\mathcal{U}}
\newcommand{\V}{\mathcal{V}}
\DeclareMathOperator{\card}{card}
\DeclareMathOperator{\tnot}{icke}
\DeclareMathOperator{\tor}{eller}
\DeclareMathOperator{\tand}{och}
\DeclareMathOperator{\lequiv}{\Longleftrightarrow}
\DeclareMathOperator{\congruent}{\equiv}

\let\stoch\relax
\DeclareMathOperator{\xor}{\oplus}


\makenomenclature
\makeindex

\begin{document}
\ifdraft{
  \fancypagestyle{plain}{
    \renewcommand{\headrulewidth}{0pt}
    \fancyhead{}
    \fancyfoot[co,ce]{\thepage}
    \fancyfoot[ro,le]{dbosk-\today}
  }
  \fancypagestyle{ams}{
    \renewcommand{\headrulewidth}{0pt}
    \fancyhead[ce]{\small\leftmark}
    \fancyhead[co]{\small\rightmark}
    \fancyhead[lo,re]{}
    \fancyhead[ro,le]{\thepage}
    \fancyfoot[ro,le]{dbosk-\today}
  }
  \pagestyle{ams}
}{}

\title[Matematik 3c]{
  Matematik 3c:\\
  En framställning med formell matematik\\
  för 2011 års ämnesplan för gymnasieskolan
}
\author{Daniel Bosk}

\frontmatter
%\begin{abstract}
%    \input{abstract.tex}
%\end{abstract}

\ifdraft{}{
    \newgeometry{rmargin=4cm,lmargin=4cm}
}
\maketitle
\ifdraft{}{
    \restoregeometry
}
%\begin{titlepage}
%   \begin{center}
%       \vskip0.1\textheight
%       \textsc{\huge \...@title}
%       \vskip0.1\textheight
%       \textsc{\large \...@author}
%       \vskip0.7\textheight
%       \textsc{\...@date}
%   \end{center}
%\end{titlepage}

\cleardoublepage
\thispagestyle{empty}
\vspace*{0.30\textheight}
\begin{flushright}
  Till min största kritiker.
\end{flushright}
\cleardoublepage

\tableofcontents
\cleardoublepage

%\chapter*{Förord till första utgåvan}
\chapter*{Förord}
\lettrine{D}{agens matematikundervisning} har fått mycket kritik de senaste
åren.
En av de mest framstående kritiseringarna är Skolinspektionens rapport 2010:13,
\emph{Undervisningen i matematik i gymnasieskolan}.
Den så kallade matematiken som de senaste årtiondena dominerat svensk skola
upplevs som räknecentrerad och eleverna lär sig genom memorering snarare än
förståelse.
Detta undervisningsmaterial syftar till att motverka denna typ av undervisning 
och inlärning.
Det ämnar att eleverna ska \emph{förstå} matematik, inte memorera metoder genom
mekaniskt räknande.

Detta verk är utvecklat att behandla det centrala innehållet i ämnesplanen 
Matematik 1c för 2011 års läroplan för gymnasieskolan.
Omfattningen av detta material är en inledning till rigorös matematik, en grund
att bygga fortsatt undervisning på.
Syftet med detta material är således inte att vara ett fördjupningsmaterial,
utan att formalisera matematiken i dagens matematikundervisning.

De gamla kursplanerna och de nya ämnesplanerna för matematik fokuserar båda
mycket på tillämpad matematik.
Det gör även de flesta matematikböcker för skolan.
I dessa framställningar kommer matematiken i sig i skymundan.
Denna text fokuserar mer på \emph{ren matematik}, anledningen är att
matematiken sällan framhålls i sin rena form och svensk tradition är tillämpad
matematik.
Kompendiet är också upplagt annorlunda och ej avsett att följa den
traditionella svenska matematikundervisningen med några givna exempel följt av
ett större antal uppgifter som föjer exemplet.
Faktum är att detta kompendium inte innehåller en enda, så kallad,
standarduppgift.
Det innehåller inte några uppgifter över huvud taget, jag har valt att kalla
dem för övningar för att komma bort från detta uppgiftsfokus.
Dessa övningar syftar till att låta läsaren öva sitt matematiska resonerande.
De ska inspirera till reflektion hos läsaren för att hjälpa denne att fördjupa 
sin förståelse för innehållet.

Materialet är framtaget med tanken att akompanieras av en undervisande lärare,
tanken är alltså inte att det ska användas för enskild läsning -- även om detta
naturligtvis är möjligt.
Forskning har visat att lärande som en social process, med diskussioner och
gemensamt resonerande, är bättre för lärande än individuellt arbete.
Tanken är alltså att innehållet ska diskuteras och reflekteras över.
Läsaren uppmuntras därför att läsa materialet och därefter diskutera det med 
någon.

Jag skrev början av detta material som en del av mitt examensarbete på
Civilingenjör- och lärarprogrammet vid Kungliga Tekniska högskolan (KTH) och
Stockholms universitet (SU) under våren 2011.
Jag vill därför rikta ett stort tack till mina handledare
Roy Skjelnes, universitetslektor i matematik vid KTH,
Lil Engström, universitetslektor emerita i matematikdidaktik vid SU, och
till Dan Laksov, professor emeritus i matematik vid KTH,
för värdefulla kommentarer och givande diskussioner under mitt arbete.

Jag har nu vidareutvecklat materialet till att täcka hela ämnesplanen, förfinat 
och utökat vissa delar.
Som lärare anser jag att information och kunskap ska vara fritt tillgängliga 
för alla, det är därför jag tillgängliggör detta verk fritt för alla att ta del 
av.
Jag publicerar det tillsammans med källkoden under licensen Creative Commons 
Erkännande-DelaLika 2.5 Sverige (CC BY-SA 2.5 SE)\footnote{%
  För att se en sammanfattning och kopia av licenstexten besök URL 
  \url{http://creativecommons.org/licenses/by-sa/2.5/se/}.
}.
Detta innebär att vem som helst får bygga vidare på denna text och publicera 
resultatet; de enda krav som föreligger är att resultatet också publiceras 
under denna (eller liknande licens) och att där står att resultatet bygger på 
detta verk.
Därför uppmuntrar jag att bygga vidare på detta arbete, och eventuella 
förbättringar inkorporerar jag mer än gärna tillbaka i detta verk.
Jag lägger även gärna till lektionsförslag eller andra förslag på tillämpningar 
som bilagor i detta verk eller hänvisar till andra tillämpningar som gjorts.

\vspace{2cm}
\begin{flushright}
  \parbox{0.5\textwidth}{
    Söråker, den 25 juni 2012\\
    \vspace{10pt}%
    Daniel Bosk
  }
\end{flushright}


\mainmatter
\ifdraft{
  \doublespacing
}{}
% main chapters
\chapter{Introduktion}
\label{ch:Intro}
\dots


%%%%%%%%%%%%%%%%%%%%%%%%%%%%%%%%%
% UPPLÄGG
%%%%%%%%%%%%%%%%%%%%%%%%%%%%%%%%%
\section{Bokens upplägg}
\dots


\part{Tal, aritmetik, algebra och geometri}
\chapter{Mer om tal}
\label{ch:mertal}
\dots

\chapter{Polynom}
\label{ch:polynom}
\dots

\chapter{Enhetscirkeln}
\label{ch:cirkeln}
\dots

\part{Samband och förändring}
\chapter{Kontinuerliga och diskreta funktioner}
\label{ch:kontdiskfunk}
\dots

\chapter{Polynomfunktioner}
\label{ch:polynomfunk}
\dots

\chapter{Derivata}
\label{ch:derivata}
\dots

\chapter{Andraderivatan}
\label{ch:andraderivatan}
\dots

\chapter{Eulers tal}
\label{ch:eulerstal}
\dots

\chapter{Extremvärdesproblem}
\label{ch:extremvarde}
\dots

\chapter{Integraler}
\label{ch:integral}


% appendices
\appendix
\ifdraft{
  \doublespacing
}{}
%\include{kursplanering}

\backmatter
\ifdraft{
  \singlespacing
}{}
\printbibliography
%\cleardoublepage
%\listoffigures
%\cleardoublepage
%\listoftables
%\cleardoublepage
%\printnomenclature
\cleardoublepage
\printindex
\cleardoublepage


\clearpage
\end{document}
